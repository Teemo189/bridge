\documentclass[12pt,oneside,a4paper]{book}%
\usepackage[T1]{fontenc}%
\usepackage[utf8]{inputenc}%
\usepackage{lmodern}%
\usepackage{textcomp}%
\usepackage{lastpage}%
\usepackage{wallpaper}%加页面背景
\usepackage{geometry}%
\geometry{left=2.5cm,right=2.5cm,top=2.5cm,bottom=2.5cm}%
\usepackage[heading=true]{ctex}%heading=true参数,这个是必须要有的%
\usepackage{multirow}%
\usepackage{setspace}%
\usepackage{lscape}%
\usepackage{float}%
\usepackage{makecell}%
\usepackage{caption}%
\usepackage{enumitem}%
\usepackage{pifont}%
\usepackage{tocloft}%
\usepackage{titletoc}%
\usepackage{fancyhdr}%
\usepackage{adjustbox}%
\usepackage{amssymb}
\usepackage{mathrsfs}
\usepackage{amsmath}
\usepackage{color}%
\usepackage{tabu}%
\usepackage{tikz}%
\usetikzlibrary{positioning,shapes,arrows}%流程图 
\usepackage{hhline}%
\usepackage[%dvipdfm,  %pdflatex,pdftex这里决定运行文件的方式不同
pdfstartview=FitH,
CJKbookmarks=true,
bookmarksnumbered=true,
bookmarksopen=true,
colorlinks, %注释掉此项则交叉引用为彩色边框(将colorlinks和pdfborder同时注释掉)
pdfborder=001,   %注释掉此项则交叉引用为彩色边框
linkcolor=black,
anchorcolor=black,
citecolor=green]{hyperref}%
\usepackage{etoolbox}%
\usepackage{seqsplit}%

\BeforeBeginEnvironment{longtabu}{\vspace{-1mm}}
\AfterEndEnvironment{longtabu}{\vspace{-1mm}}
\BeforeBeginEnvironment{figure}{\vspace{-1mm}}
\AfterEndEnvironment{figure}{\vspace{-1mm}}
\BeforeBeginEnvironment{spacing}{\vspace{-5mm}}
\AfterEndEnvironment{spacing}{\vspace{-1mm}}
\BeforeBeginEnvironment{center}{\vspace{-1mm}}
\AfterEndEnvironment{center}{\vspace{-5mm}}
%\renewcommand{\theequation}{\arabic{section}-\arabic{subsection}-\arabic{equation}}%
%
\usepackage{longtable}%
\usepackage{tabu}%
\usepackage{booktabs}%
\usepackage{tocloft} %模板中用了subfigure,不加此选项会产生冲突
\renewcommand{\cftpartleader}{\cftdotfill{\cftdotsep}} % 给 parts 加点
\renewcommand{\cftchapleader}{\cftdotfill{\cftdotsep}} % 给 chapters 加点
%\renewcommand{\cftsecleader}{\cftdotfill{\cftdotsep}} % 给 sections加点

\newcommand\mydot[1]{\scalebox{#1}{\hss$\cdot$\hss}}
\renewcommand\cftdot{\mydot{.8}} % change the size of dots

\makeatletter
\newcommand{\thickhline}{%
	\noalign {\ifnum 0=`}\fi \hrule height 1pt
	\futurelet \reserved@a \@xhline
}
\newcolumntype{"}{@{\hskip\tabcolsep\vrule width 1pt\hskip\tabcolsep}}
\makeatother

\begin{document}%
	\normalsize%
\renewcommand{\thefigure}{\arabic{chapter}.\arabic{section}-\arabic{figure}\,\,}%
\renewcommand{\thetable}{\arabic{chapter}.\arabic{section}-\arabic{table}}%
\renewcommand{\theequation}{\arabic{chapter}.\arabic{section}--\arabic{equation}}%	
	\renewcommand{\symbol}[0]{\hspace{-0.1mm}{\tt{\#}}\hspace{-0.1mm}}%
	\captionsetup[table]{  font={small,bf},belowskip=5pt,aboveskip=5pt}%设置表标题样式%
	\captionsetup[figure]{ font={small,bf},belowskip=5pt,aboveskip=5pt} %设置图标题样式,labelsep=space,定义标题序号与标题间的符号,space去掉点%period加点%不加space、period这两个就是冒号%
	%定义标题样式 section,subsection等%

	\CTEXsetup[format=\heiti\centering\bfseries\zihao{2},number ={\chinese{part}},beforeskip={200pt},afterskip={1ex plus 3pt minus 3pt}]{part}% 
	\CTEXsetup[name={},format=\songti\raggedright\bfseries\zihao{3},number ={\arabic{chapter}},beforeskip={-30pt},afterskip={1ex plus 3pt minus 3pt}]{chapter}% 
	\CTEXsetup[format=\songti\raggedright\bfseries\zihao{4},number ={\arabic{chapter}.\arabic{section}},beforeskip={5pt},afterskip={1ex plus 3pt minus 3pt}]{section}% 
	\CTEXsetup[format=\songti\flushleft\bfseries\zihao{-4},number ={\arabic{chapter}.\arabic{section}.\arabic{subsection}},beforeskip={5pt},afterskip={1ex plus 3pt minus 3pt}]{subsection}% 定义章节标题样式 subsection%
	\CTEXsetup[format=\songti\flushleft\bfseries\zihao{-4},number ={\arabic{chapter}.\arabic{section}.\arabic{subsection}.\arabic{subsubsection}},beforeskip={5pt},afterskip={1ex plus 3pt minus 3pt}]{subsubsection}% 定义章节标题样式 subsubsection%%
	\newdimen\doublelineskip % 两横线间的距离
	\setlength\doublelineskip{2pt}
	\newpage%
	
	
\makeatletter %双线页眉
\def\headrule{{\if@fancyplain\let\headrulewidth\plainheadrulewidth\fi%
		\hrule\@height 0.4pt \@width\headwidth\vskip1pt%上面线为1pt粗
		\hrule\@height 1.3pt\@width\headwidth  %下面0.5pt粗
		\vskip-2\headrulewidth\vskip-2pt}      %两条线的距离1pt
	\vspace{10mm} %双线与下面正文之间的垂直间距
}    
\makeatother
  
	\fancypagestyle{plain}{
		\fancyhf{}
		\setlength\headheight{25pt}
		\fancyhead[L]{\zihao{5}{2021年吉高速集团运营高速公路桥梁定期检查报告}}
		\fancyhead[R]{\zihao{5}{报告综述}}
		\fancyfoot[C]{第\thepage 页 }
		%	\renewcommand{\footrulewidth}{1pt}
		}%

\chapter{第一章}
\section{第一章第一节}
钢筋开始锈蚀阶段索经历的时间可按式\ref{equ:1.1.1}和式\ref{equ:1.1.2}计算,并满足下列要求:

\begin{equation}\label{equ:1.1.1} 
t_{i}=(\frac{k——{C1}}{C})^{2}
\end{equation}

\begin{equation}\label{equ:1.1.2} 
	k_{C1}=2\sqrt{D}erf^{-1}(1-\dfrac{C\cdot\gamma}{C_{T}})
\end{equation}

\section{第一章第二节}
\section{第一章第三节}

钢筋开始锈蚀阶段索经历的时间可按式\ref{equ:1.3.1}和式\ref{equ:1.3.2}计算,并满足下列要求:
\begin{equation}\label{equ:1.3.1} 
	t_{i}=(\frac{k——{C1}}{C})^{2}
\end{equation}

\begin{equation}\label{equ:1.3.2} 
	k_{C1}=2\sqrt{D}erf^{-1}(1-\dfrac{C\cdot\gamma}{C_{T}})
\end{equation}
\section{第一章第四节}
\chapter{第二章}
\section{第二章第一节}
\section{第二章第二节}
\section{第二章第三节}
\section{第二章第四节}
\end{document}